\chapter{Overview} 
	\label{overview}
	The goal of the project is to examine how the scalability of the Refinery web-application could be improved. 
	The scalability of an application is basically allowing our application to handle bigger loads and handle more simultaneous connections.
	This can mean reducing the operating costs and / or having better response times for each requests that the users of the application make. 

	As such, the current
	infrastructure has to be understood, and only then can ways be found, how improvements can be made.

	Scalability can be greatly improved via the modification of the backend architecture of an application, so that is naturally what
	I am going to describe in greater detail. 

	After examining the current infrastructure and implementation of the graph generator, 
	I will go into detail about the different approaches and ideas of how the backend could be restructured,
	so that we can achieve our goal in mind: improved scalability. 

\section{Frontend} \label{overviewfrontend}
	The frontend of the Refinery web application is implemented using the React library. As a result, most of it is written in Typescript.
	It basically works as a text editor, where the users can define the rules for their graph models, and see whether
	their rules are semantically / syntactically correct. If so, they can generate a model based on their defined rules
	via the press of a button, which sends the states of the text editor to the backend. The generation results 
	are shown on the frontend.

\section{Backend} \label{overviewbackend}
	The backend is a Java Jetty application. The clients connect to this server via WebSocket. 
	In the current implementation,
	this connection is used to send live information to the editor server, 
	regarding the state of the user-defined model.
	This state is based on what the user of the application wrote in the text editor in the
	rule-defining partial modelling language of Refinery. After each modification, only the differences
	from the last saved state are sent to the server.

	The server uses XText for keeping track of the state of the model. It is a library provided
	by Eclipse, and it makes the creation of domain specific languages (such as the one, used for Refinery)
	much easier. The most important usecase for XText, is the saving and verification of the text, 
	written in the editor.

	When a client (= an instance of the Refinery Frontend) makes a connection to backend server, a WebSocket 
	connection is made. Through this open WebSocket connection, the client continously sends JSON messages to
	the server.

	First, the initial update of the text is sent to the server.
	This inital update contains an example model, which also showcases for the user, how the language can be used
	for defining a graph problem. 
	
	After this initialization, only differences are sent to the backend, so that 
	the edits made by the user can also be applied to the state on the backend session. 
	After each user edit/modification, the
	validity of the model is verified by the server: both semantically and syntactically, and sent back to the client.

	If no action is made by the user, the connection between the client and the server
	is kept alive via the use of the ping and pong messages of WebSocket.

	If the user finished the creation of the graph model and the generate button is pressed 
	on the client-side, a message containing the generation service request is sent to the server 
	The server then proceeds to generate the model based on the user input and sends the belonging 
	states, results and metadata to the client.

	This model generation may take a significant amount of time and many computing resources. 
	If vast amount of users were to use 
	the service at the same time, the increased generation times could lead to an unpleasing user experience and 
	new user connections might also be very slow. 
	
	As a result it would be wise to restructure the Refinery infrastructure in a way, 
	where the generation happens on a separate server,
	acting like a microservice. This way, the load would be taken off from 
	the basic backend of Refinery, and the newly created microservice would be the one responsible for
	the creation of the model.
	Our task is the creation of this generator microservice.

\section{Requirements} \label{requirements}
Now that we have an idea, of what we have to perform to enhance the scalability of Refinery, we have to set some requirements for the
restructuring of the backend architecture and for the creation of the GeneratorServer service. 

\textbf{The requirements for the restructuring of Refinary should be as follows:}
\begin{enumerate}
        \item \label{requirementscale} \textbf{The backend of the application should be scalable.} This should be done, so that under increased usage 
		the user experience should not be compromised. This should be achieved via cloud infrastructure scaling.
		If more resources were allocated for the backend services, the response times should improve. Furthermore, the
		architecture of the backend should be implemented with scalability in mind.
        \item \label{requirementcancel} \textbf{The client should be able to cancel the model generation.} If the user reconsiders their need of the 
		generated model, it should be a possibility that they can cancel it. Cancelling a model generation should 
		free up server-side resources (thus, also increasing scalability).
		\item \label{requirementstate} \textbf{The client must be able to show the state of the model generation.} By this, the client-side of the application can showcase
        where the state of the model generation is at. Potential errors during the generation can also be showcased via the use of the state representation.
        \item \label{requirementtimeout} \textbf{There must be a timeout set for the model generation.} During the unlikely event of a bug in the generation implementation, 
		a timeout can limit the duration of the model generation process. This also limits heavy resource usage durations. Furthermore, if the connection would be 
		permanently lost with the frontend during the generation, the timeout could stop the no longer needed generation.
		\item \label{requirementseparateclient} \textbf{(Optional) The implementation should allow the creation of other type of cliens other than web browser-based clients.} In the future
		the application could be extended to run as a separate service. The model generations could also be used by other developers, leveraging an API for the
		generations.
		\item \label{requirementdeployment} \textbf{(Optional) It should be possible to deploy the editor and the generator together.} 
		This should improve the overall development experience of the application.
		By deploying the two services together, the live infrastructure can be simulated on the local environment.
\end{enumerate}

