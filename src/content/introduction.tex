%----------------------------------------------------------------------------
\chapter{\bevezetes}
%----------------------------------------------------------------------------

Graph analysis problems can be approached through partial modeling. By leveraging
formal methods of partial modeling, these problems can be verified using automatically
generated test data derived from the partial model definitions.

Refinery \cite{refinery} is a cloud-based "Graph Solver as a Service" web application. It supports problem
definition for problems which can be mathematically retrieved to a graph-related problems.
The application has an editor, which constantly checks for the correctness of our problem definition:
both semantically and syntactically. The editor uses the custom-made partial modeling language of Refinery.

Users of the application can generate a model, representing their defined graph problems. The generation of 
said model can be quite resource heavy and may take several seconds to perform. 
This generation is currently performed in the same backend server, towards which the frontend clients of the application
connect and communicate with. 

Under increased usage from the users, new connections made to the backend server might not be possible because of 
the increased workloads on the server. Furthermore, response times might make the user experience worse for the users, due to the increased response
times or timeouts.

This issue should only arise, under extreme usage-circumstances. The deployed infrastructure in production already mitigates this issue by 
the use of auto scaling and load balancers. However, the model generation is still done on the same server as the backend server of 
the application.

The scope of my thesis is the designing and implementation of a generator service, which can be integrated into the workflow of the model generation.
By implementing a generator service, it is hoped, that the response times under heavy usage can improve. All of this should be performed by 
putting extra care into the scalabity of the application. The costs associated with the improvements should also be considered.

In Chapter \ref{Background}, I introduce the technologies, that are already or could be used for the 
implementation of the Refinery web application.
This is done with extra attention put towards the model generation part of the application.

In Chapter \ref{overview}, I go over the current implementation of the application. I describe the frontend and the current backend of the application.
At the end, I set requirements for the modification of the application, which should be considered for all of the latter parts of the thesis.

In Chapter \ref{considerations}, I list the possible implementations for the generator service. The advantages and disadvantages of all implementation 
possibilities are considered both architecturally and infrastructurally. I compare the possibilities and draw a conclusion, of what should be implemented.

In Chapter \ref{Implementation}, I implement the architectural and the infrastructural changes that were designed in Chapter \ref{considerations}.

In Chapter \ref{Evaluation}, I evaluate how the changes implemented in chapter \ref{Implementation} satisfy the requirements made 
in section \ref{requirements}. I benchmark the new implementation compared to the old implementation. I draw a conclusion, how succesful the 
project was.

Last, but not least, in Chapter \ref{Conclusion}, I summarize the results and desribe the future improvements that can or need to be 
done in the future.
