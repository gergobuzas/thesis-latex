\pagenumbering{roman}
\setcounter{page}{1}

\selecthungarian

%----------------------------------------------------------------------------
% Abstract in Hungarian
%----------------------------------------------------------------------------
\chapter*{Kivonat}\addcontentsline{toc}{chapter}{Kivonat}
	A gráf analízisos feladatokat a parciális modellezésre lehet visszavezetni. 
	A parciális modellezés formális módszereivel lehetőség nyílik egy adott feladat megoldásainak ellenőrzésére, releváns tesztadatok automatikus generálásával, amelyeket a parciális modell definícióira alapozunk.

	Számos eszköz elérhető a parciális modellek elemzésére, azonban a használatukhoz szükséges fejlesztőkörnyezet beállítása a legtöbb felhasználó számára nehézséget okozhat. Emellett ezek az elemzési módszerek gyakran olyan erőforrásigénnyel járnak, amely meghaladja a személyi munkaállomások kapacitását.	

	Egy felhőalapú környezetben futtatható parciális modellezési eszköz használata meg
	oldást jelent a számítási kapacitás hiányára, és kiküszöböli a felhasználók számára a
 	szükséges konfigurációs terheket.

	A Refinery egy parciális modellező program, amely két fő komponensből áll: egy parciális
	nyelvi szerkesztőből és egy gráf megoldóból.

	A szakdolgozatom végső célja, hogy megvizsgáljam a Refinery gráf-modell generátor webalkalmazás jelenlegi infrastrukturális megvalósítását, és feltárjam azokat a területeket, ahol javításokra van lehetőség (legyen szó költségekről, válaszidőkről, rendelkezésre állásról vagy fenntarthatóságról).

	A jelenlegi infrastruktúra bemutatását követően megvizsgálom, hogyan lehetne javítani a skálázhatóságát,
	 részletezve az egyes megvalósítási lehetőségek előnyeit és hátrányait, 
	 valamint összehasonlítva, hogy ezek mennyire felelnek meg a dolgozat előtt meghatározott követelményeknek.

	A végső megoldást az Amazon AWS infrastruktúráján, az Elastic Container Services-t használva 
	valósítottam meg.

	A megoldásom értékeléséhez Python szkriptekkel végeztem teszteket, 
	amelyek azt vizsgálták, hogyan javultak a válaszidők több egyidejű generálási kérés esetén.


\vfill
\selectenglish


%----------------------------------------------------------------------------
% Abstract in English
%----------------------------------------------------------------------------
\chapter*{Abstract}\addcontentsline{toc}{chapter}{Abstract}
	Graph analysis problems can be approached through partial modeling. 
	By leveraging formal methods of partial modeling, these problems can be verified using 
	automatically generated test data derived from the partial model definitions. 

	While various tools exist for partial model analysis, setting up the necessary development 
	environment can be challenging for many users. Moreover, the computational demands of 
	these methods often exceed the capacity of personal workstations.

	A cloud-based partial modeling tool can address these challenges by providing the required 
	computational resources and eliminating the complexity of local configuration. 
	
	Refinery 
	is an example of such a tool, comprising two primary components: a partial language editor and a graph solver.

	The end goal of my thesis, is to examine the current infrastructural implementation of the Refinery graph-model
	generator web application, and find ways where it can be improved (whether it be related to costs, response times,
	availability or maintainability).

	After describing the current infrastructure, I look into ways, how the scalability can be improved,
	and go into detail regarding the pros and the cons of each implementation. I compare how each of them
	satisfy the specifications that had been made before the thesis.

	The final implementation is deployed using Amazon's AWS infrastructure and Elastic Container Services. 
	
	For evaluating my solution, testing was performed via Python scripts
	which checked how response times improved for multiple simultaneous generation requests.



\vfill
\cleardoublepage

\selectthesislanguage

\newcounter{romanPage}
\setcounter{romanPage}{\value{page}}
\stepcounter{romanPage}